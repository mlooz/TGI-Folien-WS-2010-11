\documentclass{article}
\usepackage[utf8]{inputenc}
\usepackage{tikz}
\usepackage{amsmath}
\usepackage{ifthen}
\usepackage[top=4cm,bottom=3cm,left=3cm,right=3cm]{geometry}
\usetikzlibrary{automata}
\newboolean{sol}
\newboolean{ex}


%%%% Config here

% Show excercises
\setboolean{ex}{true}
% Show solutions
\setboolean{sol}{true}
%%%

\usepackage{float}
\newcommand{\aufgaben}[2]{
\ifthenelse{\boolean{ex}}{
\ifthenelse{\boolean{sol}}{
	\subsection{Aufgaben}
}{}}{}
\ifthenelse{\boolean{ex}}{#1}{}
\ifthenelse{\boolean{ex}}{
\ifthenelse{\boolean{sol}}{
	\subsection{Lösungen}
}{}}{}
\ifthenelse{\boolean{sol}}{#2}{}
}

\begin{document}
\ifthenelse{\boolean{ex}}{
	\ifthenelse{\boolean{sol}}{
		\title{Übungsaufgaben 2 mit Lösung}
	}{
		\title{Übungsaufgaben 2}
	}
}{
	\ifthenelse{\boolean{sol}}{
		\title{Lösungen 2}
	}{
		\title{Leeres Blatt 2}
	}
}
\author{Simon Stroh und Moritz von Looz}
\maketitle
\section{Chomsky Normalform / CYK}
\aufgaben{.
\begin{enumerate}
\item Gib eine Grammatik in Chomsky Normalform an, die die Sprache $L_1 = \{a^nb^n | n \geq 1\}$ erzeugt
\end{enumerate}
}{.
\begin{enumerate}
\item Die Grammatik sei durch die Produktionen gegeben:
\begin{align*}
S&\rightarrow AB|AX\\
X&\rightarrow SB\\
A&\rightarrow a \\
B&\rightarrow b \\
\end{align*}
\end{enumerate}
}
\section{Ogdens und Pumping Lemma}
\aufgaben{.
Eine Aufgabe
}{.
Eine Lösung
}
\section{Greibach Normalform und Kellerautomaten}
\aufgaben{.
Eine Aufgabe
}{.
Eine Lösung
}

\end{document}
