\documentclass{beamer}
\usetheme[deutsch]{KIT}

\KITfoot{foo bar}%TODO
\usepackage[utf8]{inputenc}
\usenavigationsymbols

\title{Theoretische Grundlagen der Informatik}
\subtitle{Tutorium}
\author{Moritz von Looz, Simon Stroh}

\institute[ITI]{Intitute für Theoretische Informatik}

%TODO add something for \TitleImage[height=\titleimageht]{Bilder/bildwand.jpg}

% Based on http://piratepad.net/UWaMSSxLt6
\begin{document}

\begin{frame}
  \maketitle
\end{frame}

\begin{frame}
  \frametitle{Übersicht}
  \tableofcontents
  %TODO: irgendwie sieht die struktur noch nicht soooo gut aus, evtl überarbeiten
\end{frame}

\section{Organisatorisches}
\begin{frame}
	\frametitle{Organisatorisches}
	\begin{itemize}
		\item \textbf{Abgabe:} \emph{Handschriftlich} in Zweiergruppen.
		\item \textbf{Schein:} 
		\begin{itemize}
			\item Klausurbonus
			\item Wahscheinlich ein Notenschritt
			\item Ab 50\% der erreichbaren Punkte
		\end{itemize}
		\item \textbf{Tutoriumsmaterial:} http://supercoolernamehier.de/ %TODO
		%TODO: Liste für emailadressen rumgeben?
	\end{itemize}
\end{frame}
\section{Endliche Automaten und reguläre Sprachen}
%TODO moritz
\subsection{Formale Sprachen}
\begin{frame}
	\frametitle{Kurze Wiederholung: Formale Sprachen}
	Eine \emph{Formale Sprache} $L$ ist eine Teilmenge aller Wörtern über einem endlichen Alphabet $\Sigma$.\\[0.3cm]
	Beispiele:
	\KITframe[yes the background would be nice in gray, thanks!]
		{\parbox{\textwidth}{\begin{itemize}
			\item $\Sigma = \{ 0, 1 \}, L = \{w11z\,|\,w,z \in \Sigma^*\}$
			\begin{itemize}
			\item Die Menge aller Wörter die ''11'' enthalten.
			\end{itemize}
		\end{itemize}}
	}\\[0.2cm]
Im Allgemeinen kann man Formale Sprachen sehr frei Angeben: 
	\KITframe[Yeah, I think I'll stick to gray as a background color. Thanks again!]
		{\parbox{\textwidth}{\begin{itemize}
			\item $\Sigma = \{ 0, 1 \}, L = \{w|\,w \in \Sigma^*, w \mbox{ hat eine gerade Anzahl an $1$en)}\}$
			\begin{itemize}
			\item Die Menge aller Wörter die eine gerade Anzahl an Einsen enthalten.
			\end{itemize}
		\end{itemize}}
	}
\end{frame}
\subsection{Deterministische endliche Automaten}
\begin{frame}
\frametitle{Deterministische Endliche Automaten}
        Ein deterministischer endlicher Automat $M$ ist ein 5-Tupel
        \[
        M= (Q,\Sigma,\delta,S,F).
        \]
        \begin{itemize}
        \item $Q$:  endliche Zustandsmenge
        \item $\Sigma$:    endliches Alphabet
        \item $\delta$:   Zustandsübergangsfunktion $Q\times \Sigma \rightarrow Q$
        \item $S$:   Startzustand $\in Q$
        \item $F$:   Endzustandsmenge $\subseteq Q$
        \end{itemize}
\end{frame}
\subsection{Nichtdeterministische endliche Automaten}
\begin{frame}
\frametitle{Nichtdeterministische Endliche Automaten}
        Ein nichtdeterministischer endlicher Automat $M$ ist ein 5-Tupel
        \[
        M= (Q,\Sigma,\delta,S,F).
        \]
        \begin{itemize}
        \item $Q$:  endliche Zustandsmenge
        \item $\Sigma$:    endliches Alphabet
        \item \textcolor{red}{$\delta$:   Zustandsübergangsfunktion $Q\times (\Sigma \cup \varepsilon) \rightarrow 2^Q$}
        \item $S$:   Startzustand $\in Q$
        \item $F$:   Endzustandsmenge $\subseteq Q$
        \end{itemize}
\end{frame}
\subsection{Reguläre Ausdrücke}
%TODO moritz
\section{Konstruktion eines DEA aus einem NEA}
\subsection{Eliminieren von epsilon-Übergängen}
%TODO simon
\subsection{Potenzmengenkonstruktion}
%TODO simon
%regex 2 NEA

\end{document}

%TODO Liste

Simon
 - Aufgaben reinmachen
 - NEA
 - Potenzmengendings
 
